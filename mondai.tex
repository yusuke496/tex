\documentclass[11pt,a4paper]{jsarticle}
%
\usepackage{amsmath,amssymb}
\usepackage{bm}
\usepackage{graphicx}
\usepackage{ascmac}
%
\setlength{\textwidth}{\fullwidth}
\setlength{\textheight}{40\baselineskip}
\addtolength{\textheight}{\topskip}
\setlength{\voffset}{-0.2in}
\setlength{\topmargin}{0pt}
\setlength{\headheight}{0pt}
\setlength{\headsep}{0pt}
%
\newcommand{\divergence}{\mathrm{div}\,}  %ダイバージェンス
\newcommand{\grad}{\mathrm{grad}\,}  %グラディエント
\newcommand{\rot}{\mathrm{rot}\,}  %ローテーション
%

\title{円周率のいろは}
\author{yusuke496}
\date{\today}
\begin{document}
\maketitle

\section*{第1問}
円と円周率の定義を述べよ

\section*{第2問}
円周率は定数である。これを\pi で表す。次式を証明せよ。

\begin{eqnarray}
\int_{-1}^{1}\frac{1}{\sqrt{1-x^2}}dx=\pi \nonumber
\end{eqnarray}

なお、上式の左辺は積分可能である事は用いてよいものとする。

\section*{第3問}
第2問の結果を用いて、半径1の円の面積が\pi となることを示せ。
  
\end{document}