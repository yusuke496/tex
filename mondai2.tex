\documentclass[11pt,a4paper]{jsarticle}
%
\usepackage{amsmath,amssymb}
\usepackage{bm}
\usepackage{graphicx}
\usepackage{ascmac}
%
\setlength{\textwidth}{\fullwidth}
\setlength{\textheight}{40\baselineskip}
\addtolength{\textheight}{\topskip}
\setlength{\voffset}{-0.2in}
\setlength{\topmargin}{0pt}
\setlength{\headheight}{0pt}
\setlength{\headsep}{0pt}
%
\newcommand{\divergence}{\mathrm{div}\,}  %ダイバージェンス
\newcommand{\grad}{\mathrm{grad}\,}  %グラディエント
\newcommand{\rot}{\mathrm{rot}\,}  %ローテーション
%

\title{自然対数のABC}
%\author{yusuke496}
\date{\today}
\begin{document}
\maketitle

必要ならヒント見ながらやってみてね。\\

次の極限は収束する。

\begin{eqnarray}
    \lim_{t \rightarrow \infty}\left( 1+\frac{1}{t} \right)^t \nonumber    
\end{eqnarray}

この値を$e$と置く。

\section*{第1問}
次式を証明せよ。
\begin{eqnarray}
    \frac{d}{dx}\left( \log_e x \right) = \frac{1}{x} \nonumber
\end{eqnarray}

\section*{第2問}
$a \in \mathbb{R}$に対して次式を証明せよ。
\begin{eqnarray}
    \frac{d}{dx}\left( a^x \right) = \log_e a \times a^x \nonumber
\end{eqnarray}

\section*{第3問}
第2問の結果を用いて、次式を証明せよ。
\begin{eqnarray}
    \lim_{h \rightarrow 0} \frac{e^h-1}{h} = 1 \nonumber
\end{eqnarray}

\end{document}