\documentclass[11pt,a4paper]{jsarticle}
%
\usepackage{amsmath,amssymb}
\usepackage{bm}
\usepackage{graphicx}
\usepackage{ascmac}
%
%\setlength{\textwidth}{\fullwidth}
\setlength{\textheight}{40\baselineskip}
\addtolength{\textheight}{\topskip}
\setlength{\voffset}{-0.2in}
\setlength{\topmargin}{0pt}
\setlength{\headheight}{0pt}
\setlength{\headsep}{0pt}
%
\newcommand{\divergence}{\mathrm{div}\,}  %ダイバージェンス
\newcommand{\grad}{\mathrm{grad}\,}  %グラディエント
\newcommand{\rot}{\mathrm{rot}\,}  %ローテーション
%

\title{難しいけど解けそうで解けない気がするけどでも解ける方程式}
%\author{yusuke496}
\date{\today}
\begin{document}
\maketitle

\section*{第1問}

\begin{eqnarray}
    (x-1)(x^2+x+1)
\end{eqnarray}

を展開せよ。
(90点)

\section*{第2問}

\begin{eqnarray}
    x^4+3x^3+x^2-2=0
\end{eqnarray}

を複素数の範囲で解け。
(10点)

\newpage
\section*{解答例}
第1問は分配法則から逐一計算すると
\begin{eqnarray}
    x^3-1 \nonumber
\end{eqnarray}
となる。
第2問は頑張って次の式変形を思いつくと簡単に解ける(もちろん因数定理やその発展版を用いることもできる)。

\begin{eqnarray}
    x^4+3x^3+x^2-2&=&x^2(x^2+x+1)+2(x^3-1) \nonumber \\ 
    &=&x^2(x^2+x+1)+2(x-1)(x^2+x+1) \nonumber \\
    \label{hoge}
    &=&(x^2+x+1)(x^2+2x-2)
\end{eqnarray}

式\ref{hoge}から
\begin{eqnarray}
    \label{hoge1}
    x^2+x+1&=&0\\
    \label{hoge2}
    x^2+2x-2&=&0
\end{eqnarray}
式\ref{hoge1}、 \ref{hoge2}を解くと、

\begin{eqnarray}
    x=-1\pm \sqrt{2},\ \frac{-1\pm \sqrt{3}i}{2} \nonumber
\end{eqnarray}
を得る。ここで$i$は$\sqrt{-1}$である。

\end{document}